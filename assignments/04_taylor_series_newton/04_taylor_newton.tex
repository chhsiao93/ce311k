\documentclass[a4paper,12pt]{article}
\usepackage{graphicx}
\usepackage[left=30mm, right=30mm, top=30mm, bottom=35mm]{geometry}
\usepackage{amsmath}
\usepackage{siunitx}
\usepackage{fancyhdr}
\usepackage{url}
\pagestyle{fancy}
%-------------------------------------------------------------------------------
\lhead{\textbf{Fall 2019}}
\rhead{\textbf{CE311K Intro to Computer Methods}}
\cfoot{\thepage}
%-------------------------------------------------------------------------------

\begin{document}
\begin{centering}
	\textbf{
		Assignment 04: Taylor series and Newton Raphson
	}
\end{centering}


Note: Please upload your solution as an ipynb file to the Canvas page.

\vspace{1em}
 
 The purpose of this assignment is to develop your skills in creating approximate functions using Taylor series and find roots of functions using the Newton's iterations.
 
\begin{enumerate}
	\item Write the Taylor's series expansion of the function $f(x)
	= sin(ax)$ 
	near $x = 0$,
	where $a \ne 0$ is a known constant. Write a Python function to compute the approximate solution. Compute the relative and absolute errors using $math.sin(ax)$ as the exact answer for 3 and 5 terms. Also compute the truncation error at each iteration.
	
%	\item Write the Taylor's series expansion of the function $f(x, y)
%	= sin(x)cos(y)$ 
%	near the point $(2, 2)$. Write a Python function to compute the approximate solution. Compute the relative and absolute errors using the  $math$ module as the exact answer for the first 3 terms. Also compute the truncation error at each iteration.
	
	\item Determine the root of $f(x) = x - 2e^{-x}$ by:
	\begin{enumerate}
		\item Using the bisection method. Start with $a = 0$ and $b = 1$, and compute the error at the end of the first three iterations.
		\item Using the Newton's method, Start at $x_0 = 1$ and compute the roots at the end of the the first three iterations. Compare the error against the bisection approach.
	\end{enumerate}

	\item Using the Newton-Raphson iteration find the root of the equation $\sqrt{x} + x^2 = 7$. With an initial guess of $x = 7$ compute the number of iterations required for the error in the root to be below $1.0e^{-6}$.
\end{enumerate}

\end{document}

